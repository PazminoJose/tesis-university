En \cite{benitezalvarezImplementingComprehensiveCitizen2019} se señala que en 2023 Ecuador se
ha visto envuelto en una gran cantidad de incidentes violentos, empezando
desde asaltos, robos, secuestros, violaciones hasta muertes violentas, causando preocupación
en la ciudadanía. En ese contexto, en \cite{alvarezCompendioEconomicoSocial2022} se señala
cómo la mayor cantidad de reportes a la policía proceden de denuncias administrativas.
Estas denuncias no corresponden al total de delitos reales que han ocurrido, debido al hecho de
que no todos los ciudadanos recurren a denunciar formalmente los delitos de los cuales han sido víctimas.
% Tomando en cuenta que desde el año 2008 el Ecuador se ha enfrentado a una
% crisis en relación al crimen e inseguridad llegando a niveles inquietantes de hasta un 18.9\%
% por cada 100.000 habitantes que se han visto afectados por la delincuencia.
Según \cite{castroaniyarPaintingsCrimeComposed2019} esto causa que los datos obtenidos no sean
completamente integrales y precisos en cuanto al lugar y momento en que ocurrieron los incidentes, y
a su vez, dificultan su uso en procesos de análisis y toma de decisiones adecuadas.

\bigbreak
% Según datos registrados por el Compendio económico y social OBEST - INFORMA mediante una encuesta dirigida a todos los ciudadanos del cantón Ambato,
% se determinó que el 26.30\% fue víctima de algún tipo de siniestro entre los períodos de octubre de 2020 y marzo de 2022,
% siendo los delitos más comunes los asaltos y robos con un 33\% del total de delitos y el robo a viviendas con un 8\% del total \cite{alvarezCompendioEconomicoSocial2022}.\\

% En \cite{comisionestadisticadeseguridadciudadanayjusticiaEstadisticasSeguridadIntegral2022} según datos
% obtenidos de la Comisión Estadística de Seguridad Ciudadana y Justicia, entre los años 2021 y 2022,
% se ha notado un incremento de casi un 100\% en el número de homicidios pasando de 2496 a 4823 casos reportados
% y al menos un 25\% en el robo a personas incrementando de 25440 a 31485 incidentes registrados.

% El autor de \cite{naranjo-avalosImpactoGeorreferenciacionColaborativa2019} menciona que se debe
% tomar en cuenta la necesidad de otorgar acceso a la ciudadanía a un sistema que permita la
% recolección de información georreferenciada sobre delitos, dado que la misma información ingresada pueda
% ser validada por la propia ciudadanía, quien, en su defecto, son las víctimas de los siniestros.
% Considerando
% lo expuesto, se justifica la necesidad de crear una herramienta accesible al público y de uso intuitivo. Esta
% plataforma permitirá a los ciudadanos registrar datos georreferenciados sobre los incidentes de los que han sido
% víctimas. Además, brindará un medio para reportar emergencias en tiempo real. Este recurso se convertirá en una
% valiosa fuente de información con la cual se pueda desarrollar un modelo analítico de BI crucial para la toma de
% decisiones que contribuyan significativamente a mejorar la seguridad pública.


El autor de \cite{naranjo-avalosImpactoGeorreferenciacionColaborativa2019} menciona la importancia de proporcionar a
la ciudadanía un sistema accesible que permita recopilar información georreferenciada sobre delitos. Esta información,
una vez ingresada, podría ser validada por los propios ciudadanos, quienes, en su defecto, son las víctimas de estos
incidentes. En este contexto, se justifica la necesidad de desarrollar una herramienta pública, intuitiva y de fácil acceso.
Esta herramienta posibilitaría a los ciudadanos registrar datos georreferenciados  de incidentes que han experimentado
y también ofrecería un canal para reportar emergencias en tiempo real. La creación de esta  herramienta se vuelve crucial,
ya que se convertiría en una valiosa fuente de información. Esta base de datos georreferenciados  podría ser utilizada
para desarrollar un modelo analítico de Business Intelligence (BI) que sería fundamental para la toma de  decisiones
en materia de seguridad pública.

\bigbreak
El presente proyecto es técnicamente factible, ya que las herramientas necesarias para su desarrollo son de acceso libre y
fácil. Además, las especificaciones requeridas del sistema permiten su integración tanto en una aplicación web como móvil.
Desde una perspectiva operativa, el sistema puede ser desarrollado, ya que la recolección de datos será realizada por los
propios usuarios del sistema, quienes son ciudadanos de Ambato.
En cuanto a la factibilidad económica, el proyecto puede llevarse a cabo con costos bajos, los cuales serán asumidos por
el autor de la investigación. Esta estructura financiera asegura la independencia del proyecto al no depender de financiamiento
externo, garantizando su viabilidad económica a lo largo de su desarrollo.