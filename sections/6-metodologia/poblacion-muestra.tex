
\subsection{Población y muestra}

Al no contar con una estimación adecuada sobre la cantidad de personas requerida para estipular
la población, se optó por definir una muestra infinita con población desconocida.
\begin{itemize}
    \item n = tamaño de la muestra
    \item Z = nivel de confianza, un nivel de confianza habitual es de 95\% con una puntuación estándar de 1.96
    \item p = probabilidad de éxito o proporción esperada
    \item q = probabilidad de fracaso
    \item pq = varianza de la población, cada una equivale a 0.50, (p=q=0.50)
    \item e = error de estimación máximo aceptable, el error habitual es de 5\% con una puntuación estándar de 0.05
\end{itemize}
\[
    n=\frac{Z^2 \cdot p \cdot q}{e^2}
\]\\
\[
    n=\frac{(1.96)^2 \cdot 0.50 \cdot 0.50}{(0.05)^2}
\]\\
\[
    n=384.16 \approx 385
\]
Con un nivel de confianza del 95\% y un error de estimación máximo aceptable del 5\%, se obtiene una muestra aproximada de 385 personas. \\ \\